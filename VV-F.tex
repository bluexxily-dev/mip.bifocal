\documentclass[12pt,a4paper]{article}
\usepackage[utf8]{inputenc}
\usepackage[T1]{fontenc}
\usepackage[slovak,english]{babel}
\usepackage{geometry}
\usepackage{array}
\usepackage{longtable}
\usepackage{multirow}
\usepackage{setspace}
\usepackage{tabularx}
\usepackage{ragged2e}
\usepackage{enumitem}
\usepackage{ulem}

\geometry{a4paper, margin=2cm}
\setlength{\parindent}{0pt}
\setlength{\parskip}{6pt}
\renewcommand{\arraystretch}{1.25}

% helper for fillable line
\newcommand{\fillline}[1][6cm]{\rule{#1}{0.4pt}}
\newcommand{\smallfill}[1][3cm]{\rule{#1}{0.4pt}}

% column types
\newcolumntype{L}[1]{>{\RaggedRight\arraybackslash}p{#1}}

\begin{document}

\section*{VV-F – VECNÝ ZÁMER PROJEKTU / \\ MATERIAL INTENT OF THE PROJECT}

\begin{longtable}{|p{0.35\textwidth}|p{0.55\textwidth}|}
\hline
Názov projektu & Blockchain pre sledovateľnosť potravín \\ \hline
Zodpovedný riešiteľ & Simona Kubicseková \\ \hline
Žiadateľ & \\ \hline
Štatutárny/i zástupca/ovia žiadateľa & \\ \hline
\end{longtable}

\subsection*{1. Excelentnosť / Excellence}
\vspace{0.5em}

Projekt „Blockchain pre sledovateľnosť potravín“ navrhuje integrovaný rámec sledovateľnosti potravín s využitím blockchainovej technológie kombinovanej s IoT zariadeniami, QR/RFID označovaním a smart kontraktmi. Navrhované riešenie umožňuje monitorovanie potravinových výrobkov od pôvodu až po spotrebu v reálnom čase, čím sa minimalizuje riziko kontaminácie a zvyšuje dôvera spotrebiteľov. Očakávané výsledky sú aplikovateľné v praxi, pričom projekt spĺňa úroveň TRL 1.

Originalita projektu spočíva v kombinácii decentralizovanej blockchainovej infraštruktúry s existujúcimi národnými databázami a systémami EÚ, čím sa dosahuje transparentnosť, bezpečnosť a interoperabilita. Stav poznania vychádza z najnovších výskumov blockchainu v logistike a potravinárstve, ako aj predchádzajúcich výsledkov z pilotných implementácií.

Ciele projektu sú reálne dosiahnuteľné v rámci plánovaného harmonogramu a zahŕňajú: vývoj fungujúceho systému sledovateľnosti, školenia zamestnancov, dokumentáciu a prototyp senzorov. Metodológia zahŕňa analýzu súčasných výziev, návrh systému, integráciu IoT zariadení, pilotnú implementáciu a následné rozšírenie systému do ďalších sektorov.

Významné výstupy zodpovedného riešiteľa v posledných 5 rokoch zahŕňajú publikácie a projekty v oblasti IT a blockchainu, ktoré prispeli k rozvoju digitálnych systémov a aplikácií v logistike a potravinárstve. Expertíza Simony Kubicsekovej a jej vízia budovania výskumného tímu zabezpečujú kompetentnosť na realizáciu projektu. Mladí pracovníci a doktorandi budú aktívne zapojení do riešenia projektu, podporujúc rozvoj ľudských zdrojov vo výskume a vývoji.

\newpage

\subsection*{2. Dopad / Impact}
\vspace{0.5em}

Projekt prispeje k rozvoju poznania a aplikovaného výskumu v oblasti blockchainu a sledovateľnosti potravín, čo umožní: 

\begin{itemize}
    \item zvýšenie bezpečnosti potravín a dôvery spotrebiteľov,
    \item efektívnejšie riadenie dodávateľských reťazcov,
    \item podporu regulačného dohľadu a rozhodovania,
    \item zavedenie udržateľných postupov a zníženie potravinového odpadu.
\end{itemize}

Deklarované výsledky budú využiteľné pre Slovenskú republiku a potenciálne aj v zahraničí pri integrácii s platformami EÚ (TRACES, EFSA). Projekt má ekonomický prínos cez zvýšenie podielu pridanej hodnoty, optimalizáciu zdrojov, podporu zamestnanosti a zlepšenie kvality života spotrebiteľov. 

Plánované opatrenia na maximalizáciu dopadu zahŕňajú: školenia účastníkov, vytvorenie dokumentácie a prototypov, komunikáciu výsledkov prostredníctvom odborných publikácií a prezentácií, a popularizáciu výsledkov medzi odbornou aj širšou verejnosťou.

\newpage

\subsection*{3. Implementácia / Implementation}
\vspace{0.5em}

Projekt bude realizovaný v etapách podľa harmonogramu:

\begin{itemize}
    \item Výskum a analýza súčasných výziev v slovenských dodávateľských reťazcoch,
    \item Identifikácia regulačných požiadaviek a iniciatív EÚ,
    \item Návrh systému, registrácia a overovanie účastníkov, integrácia digitálnych nástrojov, inštalácia IoT senzorov, QR/RFID označovanie a školenie zamestnancov,
    \item Pilotná implementácia v prioritných sektoroch (mlieko a mäso) s vyhodnotením výkonnosti a nákladovej efektívnosti,
    \item Rozšírenie systému do ďalších sektorov (ovocie, zelenina, nápoje).
\end{itemize}

Projektový manažment bude zabezpečený zodpovedným riešiteľom a realizačným tímom s využitím existujúcej technickej a personálnej infraštruktúry STU. Identifikované riziká, ako technická interoperabilita a prijatie používateľov, budú zmiernené pilotnými testami, školeniami a priebežnou spätnou väzbou. 

Rozpočet projektu je adekvátny vzhľadom na plánované výstupy, pričom hlavné náklady sú alokované na mzdy, materiál, služby a implementáciu IoT a blockchain infraštruktúry. Projekt využije existujúcu laboratórnu infraštruktúru, servery a softvérové riešenia na spracovanie a vizualizáciu dát.



\vfill

\end{document}