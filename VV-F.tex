\documentclass[12pt,a4paper]{article}
\usepackage[utf8]{inputenc}
\usepackage[T1]{fontenc}
\usepackage[slovak,english]{babel}
\usepackage{geometry}
\usepackage{array}
\usepackage{longtable}
\usepackage{multirow}
\usepackage{setspace}
\usepackage{tabularx}
\usepackage{ragged2e}
\usepackage{enumitem}
\usepackage{ulem}

\geometry{a4paper, margin=2cm}
\setlength{\parindent}{0pt}
\setlength{\parskip}{6pt}
\renewcommand{\arraystretch}{1.25}

% helper for fillable line
\newcommand{\fillline}[1][6cm]{\rule{#1}{0.4pt}}
\newcommand{\smallfill}[1][3cm]{\rule{#1}{0.4pt}}

% column types
\newcolumntype{L}[1]{>{\RaggedRight\arraybackslash}p{#1}}

\begin{document}

\section*{VV-F – VECNÝ ZÁMER PROJEKTU / \\ MATERIAL INTENT OF THE PROJECT}

\begin{longtable}{|p{0.35\textwidth}|p{0.55\textwidth}|}
\hline
Názov projektu & Blockchain pre sledovateľnosť potravín \\ \hline
Zodpovedný riešiteľ & Simona Kubicseková \\ \hline
Žiadateľ & \\ \hline
Štatutárny/i zástupca/ovia žiadateľa & \\ \hline
\end{longtable}

\subsection*{1. Excelentnosť / Excellence}
\vspace{0.5em}

Projekt „Blockchain pre sledovateľnosť potravín“ navrhuje integrovaný rámec sledovateľnosti potravín s využitím blockchainovej technológie kombinovanej s IoT zariadeniami, QR/RFID označovaním a smart kontraktmi. Navrhované riešenie umožňuje monitorovanie potravinových výrobkov od pôvodu až po spotrebu v reálnom čase, čím sa minimalizuje riziko kontaminácie a zvyšuje dôvera spotrebiteľov. Očakávané výsledky sú aplikovateľné v praxi, pričom projekt spĺňa úroveň TRL 1.

Vychádza z dôkladnej analýzy súčasného stavu agropotravinárského sektora na Slovensku, ktorý je charakteristický fragmentáciou dátových systémov, obmedzenou interoperabilitou a zastaranými metódami evidencie. Hlavným vedeckým prínosom je návrh hybridného blockchainového modelu, ktorý prekonáva limity súčasných centralizovaných riešení. Tento model unikátne kombinuje:
\begin{itemize}
\item Povolenú blockchainovú sieť typu konsorcium pre overených účastníkov (ministerstvá, producenti, distribútori), čím zabezpečuje súkromie pri zachovaní transparentnosti.
\item Hybridnú architektúru ukladania dát: Kritické údaje sledovateľnosti (hashované transakcie, časové pečiatky, prevody vlastníctva) sú ukladané on-chain, zatiaľ čo rozsiahle súbory (fotografie, laboratórne správy, certifikáty) sú uchovávané off-chain v šifrovaných úložiskách s hash odkazmi. Toto riešenie optimalizuje škálovateľnosť a náklady.
\item Hlbokú integráciu s technológiami Industry 4.0: Senzory IoT, RFID štítky a QR kódy automatizujú zber dát v reálnom čase (teplota, vlhkosť, poloha), čím sa minimalizuje ľudská chyba a úmyselné falšovanie.
\item Využitie smart kontraktov pre automatizáciu obchodných procesov, ako sú platby po doručení alebo vystavovanie certifikátov kvality.
\end{itemize}

Tento  prístup zabezpečuje, že technologické riešenie je použiteľné, udržateľné a prijateľné pre všetky zúčastnené strany.





\newpage

\subsection*{2. Dopad / Impact}
\vspace{0.5em}

Projekt prispeje k vytvorenie overiteľnej metodológie pre implementáciu blockchainu v národnom meradle, ktorá môže slúžiť ako vzor pre iné krajiny V4 alebo EÚ.

Z hľadiska spoločenského dopadu projekt zásadne zvyšuje bezpečnosť potravín. Umožňuje okamžitú a presnú identifikáciu zdroja potenciálnej kontaminácie, čím sa výrazne znižuje rozsah aj ekonomická záťaž spojená so sťahovaním výrobkov z trhu, a v konečnom dôsledku sa chránia zdravie a dôvera spotrebiteľov. Táto dôvera je budovaná aj priamo, keďže koncovému zákazníkovi sú prostredníctvom jednoduchého naskenovania QR kódu poskytnuté overiteľné a komplexné informácie o pôvode, kvalite a udržateľnosti zakúpených potravín.

Z ekonomického hľadiska systém posilňuje konkurencieschopnosť slovenských poľnohospodárov a výrobcov. Transparentne preukázateľný pôvod sa v súčasnosti stáva kľúčovou konkurenčnou výhodou na náročných medzinárodných trhoch. Táto transparentnosť otvára slovenským produktom prístup k prémiovým segmentom a priamo tak prispieva k zvyšovaniu pridanej hodnoty slovenského exportu.

Deklarované výsledky budú využiteľné pre Slovenskú republiku a potenciálne aj v zahraničí pri integrácii s platformami EÚ (TRACES, EFSA). Projekt má ekonomický prínos cez zvýšenie podielu pridanej hodnoty, optimalizáciu zdrojov, podporu zamestnanosti a zlepšenie kvality života spotrebiteľov. 

Plánované opatrenia na maximalizáciu dopadu zahŕňajú: školenia účastníkov, vytvorenie dokumentácie a prototypov, komunikáciu výsledkov prostredníctvom odborných publikácií a prezentácií, a popularizáciu výsledkov medzi odbornou aj širšou verejnosťou.

Projekt aktívne podporuje udržateľnosť a environmentálnu zodpovednosť. Systém je navrhnutý tak, aby umožňoval sledovanie kľúčových environmentálnych ukazovateľov, ako je uhlíková stopa, spotreba vody alebo aplikácia pesticídov. Tým priamo napĺňa stratégie Európskej zelenej dohody (European Green Deal) a poskytuje spotrebiteľom potrebné údaje pre informovanejšie a ekologickejšie rozhodovanie. Dôležitým aspektom je aj inklúzia malých a stredných podnikov. Vývoj intuitívnych a zjednodušených rozhraní, ako sú mobilné aplikácie, umožní aj menším farmárom a producentom zapojiť sa do tohto transparentného ekosystému a získať spravodlivejší prístup na trh.



\newpage

\subsection*{3. Implementácia / Implementation}
\vspace{0.5em}

Projekt bude realizovaný v etapách podľa harmonogramu:

\begin{itemize}
    \item Výskum a analýza súčasných výziev v slovenských dodávateľských reťazcoch,
    \item Identifikácia regulačných požiadaviek a iniciatív EÚ,
    \item Návrh systému, registrácia a overovanie účastníkov, integrácia digitálnych nástrojov, inštalácia IoT senzorov, QR/RFID označovanie a školenie zamestnancov,
    \item Pilotná implementácia v prioritných sektoroch (mlieko a mäso) s vyhodnotením výkonnosti a nákladovej efektívnosti,
    \item Rozšírenie systému do ďalších sektorov (ovocie, zelenina, nápoje).
\end{itemize}

Projektový manažment bude zabezpečený zodpovedným riešiteľom a realizačným tímom s využitím existujúcej technickej a personálnej infraštruktúry STU. Identifikované riziká, ako technická interoperabilita a prijatie používateľov, budú zmiernené pilotnými testami, školeniami a priebežnou spätnou väzbou. 

Rozpočet projektu je adekvátny vzhľadom na plánované výstupy, pričom hlavné náklady sú alokované na mzdy, materiál, služby a implementáciu IoT a blockchain infraštruktúry. Projekt využije existujúcu laboratórnu infraštruktúru, servery a softvérové riešenia na spracovanie a vizualizáciu dát.



\vfill

\end{document}