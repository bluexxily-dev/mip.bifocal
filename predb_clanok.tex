\documentclass[10pt,twoside,slovak,a4paper]{article}

\usepackage[slovak]{babel}
\usepackage[IL2]{fontenc}
\usepackage[utf8]{inputenc}
\usepackage{graphicx}
\usepackage{url} 
\usepackage{hyperref} 
\usepackage{fancyhdr}

\usepackage{cite}

\pagestyle{headings}

\title{BIFOCAL\thanks{Semestrálny projekt v predmete Metódy inžinierskej práce, ak. rok 2025/26, vedenie: Ivan Kapustík}} 

\author{Simona Kubicseková, Dominik Kužma, Maroš Kvietok \\[2pt]
	{\small Slovenská technická univerzita v Bratislave}\\
	{\small Fakulta informatiky a informačných technológií}\\
	}

\date{\small 30. september 2025}


\begin{document}

\maketitle

\begin{abstract}
Dodávateľské reťazce v agropotravinárskom priemysle sú zložité a zahŕňajú veľký počet aktérov – od malých poľnohospodárov, prvospracovateľov a obchodníkov až po výrobcov produktov, distribútorov, maloobchodníkov a spotrebiteľov. Táto zložitosť prináša výzvy na udržiavanie bezpečnosti, transparentnosti a kvality potravín, ako aj zvyšovanie rizika podvodov, kontaminácie a neefektívnosti. Spotrebitelia sa čoraz viac zaujímajú o pôvod a kvalitu potravín, čo vyvoláva tlak na producentov a distribútorov, aby poskytovali transparentné a overiteľné informácie. 

Táto štúdia skúma aplikáciu technológie blockchain ako decentralizovaného rámca na zlepšenie sledovateľnosti potravín. Navrhovaný systém umožňuje sledovanie potravinových výrobkov v reálnom čase od pôvodu až po spotrebu, čo by účinne obmedzovalo prepuknutia kontaminácie a umožňovalo ľuďom vysledovať produkt v priebehu niekoľkých sekúnd namiesto týždňov. Štúdia využíva analytický prístup kombinujúci literárnu rešerš, návrh modelu blockchainového systému a prípadovú štúdiu z praxe agropotravinárskeho sektora. 

Očakávaným prínosom je zvýšenie dôvery spotrebiteľov, zníženie administratívnych nákladov a podpora udržateľných postupov v potravinárstve. Implementácia blockchainu môže tiež uľahčiť spoluprácu medzi aktérmi a zabezpečiť lepšiu transparentnosť v celom hodnotovom reťazci.
\end{abstract}

\clearpage
\tableofcontents
\thispagestyle{plain} % čistá hlavička

\clearpage
\pagestyle{plain}

\section{Úvod}

\section{Teoretický rámec} \label{teoria}

\subsection{Základné pojmy} \label{zakladne pojmy}

\subsection{Blockchain} \label{blockchain}

\subsection{Sledovateľnosť potravín (Food Traceability)} \label{sledovatelnost}

\section{Integrácia blockchainu do sledovania potravín} \label{ina}


\section{Prípadové štúdie / praktické príklady} \label{dolezita}


\section{Záver} \label{zaver} 

\bibliography{literatura}
\bibliographystyle{plain}

\end{document}
