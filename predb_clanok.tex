\documentclass[10pt,twoside,slovak,a4paper]{article}

\usepackage[slovak]{babel}
\usepackage[IL2]{fontenc}
\usepackage[utf8]{inputenc}
\usepackage{graphicx}
\usepackage{float}
\usepackage{url}
\usepackage{xurl}
\usepackage{hyperref} 
\usepackage{fancyhdr}
\usepackage{booktabs}

\usepackage{cite}

\pagestyle{headings}

\title{Blockchain pre sledovateľnosť potravín\thanks{Semestrálny projekt v predmete Metódy inžinierskej práce, ak. rok 2025/26, vedenie: Ivan Kapustík}} 

\author{Simona Kubicseková, Dominik Kužma, Maroš Kvietok \\[2pt]
	{\small Slovenská technická univerzita v Bratislave}\\
	{\small Fakulta informatiky a informačných technológií}\\
	}

\date{\small 30. september 2025}


\begin{document}

\maketitle

\begin{abstract}
Dodávateľské reťazce v agropotravinárskom priemysle sú zložité a zahŕňajú veľký počet aktérov. Spotrebitelia sa čoraz viac zaujímajú o pôvod a kvalitu potravín, čo vyvoláva tlak na producentov a distribútorov, aby poskytovali transparentné a overiteľné informácie. 

Táto štúdia sa zameriava na využitie technológie blockchain ako decentralizovaného rámca na zlepšenie sledovateľnosti potravín. Navrhovaný systém umožňuje monitorovanie potravinových výrobkov v reálnom čase od pôvodu až po spotrebu, čo by účinne obmedzovalo riziko a dopad kontaminácie potravín, pretože by umožnil rýchlu a presnú identifikáciu zdroja problému. 

Štúdia uplatňuje analytický prístup, ktorý kombinuje literárnu rešerš, návrh modelu blockchainového systému a prípadovú štúdiu z praxe agropotravinárskeho sektora. 

Očakávaným prínosom je zvýšenie dôvery spotrebiteľov, zníženie administratívnych nákladov a podpora udržateľných postupov v potravinárstve. Implementácia blockchainu môže tiež uľahčiť spoluprácu medzi aktérmi a zabezpečiť lepšiu transparentnosť v celom hodnotovom reťazci.
\end{abstract}

\clearpage
\tableofcontents
\thispagestyle{plain}

\clearpage
\pagestyle{plain}

\section{Úvod}

\section{Teoretický rámec} \label{teoria}

\subsection{Základné pojmy} \label{zakladne pojmy}

\subsection{Blockchain} \label{blockchain}

Technológia \textit{blockchain} predstavuje spôsob uchovávania dát v decentralizovanej sieti, ktorý eliminuje potrebu centrálneho sprostredkovateľa. Vznikla v roku 2008 ako súčasť návrhu kryptomeny Bitcoin, no odvtedy sa jej využitie rozšírilo do mnohých oblastí – od bankovníctva až po verejnú správu. 
\vspace{1em}

Blockchain možno v jednoduchej forme predstaviť ako reťaz blokov (ako môžeme vidieť na obrázku~\ref{fig:blockchain_architecture}), kde každý blok obsahuje súbor transakcií a odkaz (tzv. hash) na predchádzajúci blok. Tento mechanizmus zabezpečuje, že žiadny blok nemožno spätne zmeniť bez narušenia celej štruktúry reťazca. Takto sa vytvára dôveryhodný a nemenný záznam udalostí, ktorý je navyše distribuovaný medzi všetkými účastníkmi siete.

\begin{figure}[H]
    \centering
    \includegraphics[width=0.8\textwidth]{blockchain.png}
    \caption{Zjednodušená architektúra blockchainu}
    \label{fig:blockchain_architecture}
\end{figure}

Jednou z hlavných výhod blockchainu je jeho \textit{transparentnosť}. Každý účastník siete môže overiť údaje bez nutnosti dôvery v konkrétnu autoritu. Tento princíp je užitočný všade tam, kde je dôležité zabezpečiť integritu informácií – napríklad v logistike, energetike, či evidencii majetku. Naopak, nevýhodou býva energetická náročnosť niektorých mechanizmov a obmedzená rýchlosť spracovania transakcií.
\vspace{1em}

V posledných rokoch sa výskum zameriava na efektívnejšie modely, ako sú \textit{Proof of Stake} alebo \textit{Byzantine Fault Tolerance}, ktoré umožňujú rýchlejšie a ekologickejšie spracovanie údajov. Tieto prístupy robia blockchain atraktívnym aj pre oblasti mimo finančného sektora, napríklad pre zdravotníctvo či verejné registre \cite{zheng2018blockchain}.
\vspace{1em}

V súčasnosti sa blockchain skúma aj ako nástroj pre transparentnosť a dôveryhodnosť v rámci dodávateľských reťazcov, kde umožňuje bezpečné zdieľanie údajov medzi nezávislými aktérmi. Aj keď jeho implementácia prináša technické a organizačné výzvy, je považovaný za perspektívnu technológiu s potenciálom zásadne zmeniť spôsob, akým pracujeme s dátami.


\subsection{Sledovateľnosť potravín (Food Traceability)} \label{sledovatelnost}

Na poznatkoch o fungovaní technológie \textit{blockchain} je možné ďalej stavať pri jej uplatnení v konkrétnych oblastiach, kde je transparentnosť a dôveryhodnosť údajov mimoriadne dôležitá. Jednou z takýchto oblastí je \textit{sledovateľnosť potravín} (\textit{food traceability}), ktorá sa v súčasnosti stáva jednou z kľúčových tém v oblasti potravinovej bezpečnosti a riadenia dodávateľských reťazcov.
\vspace{1em}

\vspace{1em}

V posledných rokoch sa koncept sledovateľnosti potravín rýchlo rozvíja, keďže rastie dopyt po bezpečných a overiteľných produktoch. Viaceré podniky a obchodné reťazce implementujú digitálne riešenia, ktoré umožňujú presne určiť pôvod a kvalitu potravín. Napríklad využitie blockchainu v potravinovom sektore dokázalo výrazne skrátiť čas potrebný na vysledovanie pôvodu produktu z niekoľkých dní na niekoľko sekúnd. Podobné systémy testujú aj ďalšie svetové spoločnosti, ktoré sa snažia zvýšiť dôveru spotrebiteľov a eliminovať falšovanie potravín. Tieto príklady potvrdzujú, že blockchain nie je len teoretickým konceptom, ale aj praktickým nástrojom, ktorý dokáže zlepšiť rýchlosť a spoľahlivosť vysledovateľnosti \cite{tian2017food}.
\vspace{1em}

V rámci moderných riešení sa blockchain často kombinuje s technológiami Industry~4.0 a Web~3.0. Ide najmä o internet vecí (IoT), RFID, QR kódy, cloud computing a GPS. Tieto technológie umožňujú zber a prenos údajov v reálnom čase, ktoré sú následne zabezpečené a zdieľané prostredníctvom blockchainu. Zatiaľ čo niektoré pokročilé nástroje, ako umelá inteligencia, analýza veľkých dát, \textit{edge computing}, IPFS či metaverzum, sa zatiaľ používajú len zriedka, práve ich prepojenie s blockchainom predstavuje potenciál pre sofistikovanejšie a efektívnejšie riešenia sledovateľnosti.
\vspace{1em}

Zaujímavý smer predstavujú aj návrhy inteligentných systémov, ktoré kombinujú blockchain s umelou inteligenciou, cloudom a \textit{big data} na znižovanie potravinového odpadu. Takéto prístupy ukazujú, že sledovateľnosť nemusí byť len o spätnej kontrole pôvodu, ale aj o aktívnom riadení tokov potravín, predchádzaní stratám a efektívnejšom využívaní zdrojov. V budúcnosti môžu podobné modely umožniť rýchlejšie reakcie na krízy, lepšiu udržateľnosť a optimalizáciu celého dodávateľského reťazca.
\vspace{1em}

Tieto princípy majú paralely aj v iných odvetviach, kde sa blockchain využíva na zvýšenie dôvery a automatizáciu procesov, napríklad vo financiách či logistike. Spoločným znakom je využitie distribuovaného registra a \textit{smart kontraktov}, ktoré zabezpečujú transparentnosť a automatické vynucovanie pravidiel. V potravinovom sektore to znamená, že každý krok – od farmára cez spracovateľa až po obchodníka – môže byť zaznamenaný a overený bez možnosti dodatočnej manipulácie.
\vspace{1em}

Na začiatku tohto projektu sa sledovateľnosť potravín chápe ako súbor technológií a procesov, ktoré umožňujú prepojiť údaje o pôvode, spracovaní a distribúcii potravín do jednotného, dôveryhodného systému. Cieľom je zvýšiť transparentnosť, udržateľnosť a bezpečnosť potravinového reťazca, pričom blockchain môže slúžiť ako technický základ pre tieto ciele.

\section{Integrácia blockchainu do sledovania potravín} \label{ina}

V posledných rokoch sa sledovateľnosť potravín stala základom dôvery spotrebiteľov aj dodržiavania predpisov v celej Európe. Globálne dodávateľské reťazce potravín sú čoraz komplexnejšie, a preto je schopnosť sledovať pôvod, manipuláciu a distribúciu potravinových výrobkov otázkou bezpečnosti aj konkurencieschopnosti.
\cite{Patel2023FoodSafety}.
\vspace{1em}

Slovensko, člen Európskej únie so silnou poľnohospodárskou tradíciou, čelí výzvam pri harmonizácii domácich systémov sledovateľnosti s európskymi digitálnymi a environmentálnymi politikami. Fragmentované dátové systémy, obmedzená interoperabilita medzi regionálnymi poľnohospodárskymi družstvami a zastarané postupy vedenia záznamov bránia zabezpečeniu transparentných a efektívnych dodávateľských reťazcov.
\cite{Floris2022SupplyChains_Slovakia}.
\vspace{1em}

Implementácia systému sledovateľnosti potravín založeného na blockchaine predstavuje príležitosť zvýšiť spoľahlivosť, efektívnosť a globálnu konkurencieschopnosť slovenského poľnohospodárstva. Táto časť článku navrhuje komplexný plán jeho zavedenia, ktorý zahŕňa inštitucionálne, právne, technické a prevádzkové aspekty. Cieľom je vytvoriť integrovaný rámec sledovateľnosti v súlade so stratégiou digitálnej transformácie Slovenska a iniciatívami EÚ „Z farmy na stôl“ a „Digitálna Európa“.
\cite{EUrada}.
\vspace{1em}

\textbf{Strategický cieľ} národného systému sledovateľnosti potravín na Slovensku je vytvoriť jednotnú digitálnu infraštruktúru, ktorá umožní komplexnú viditeľnosť v celom dodávateľskom reťazci – od prvovýroby až po maloobchod. Tento cieľ podporuje niekoľko kľúčových priorít:

\begin{itemize}
\item \textbf{Bezpečnosť potravín a dôvera spotrebiteľov:} Umožňuje rýchlu reakciu na kontamináciu a sťahovanie výrobkov z trhu.
\item \textbf{Konkurencieschopnosť vývozu:} Transparentný a overiteľný pôvod výrobkov je čoraz viac požiadavkou medzinárodných trhov.
\item \textbf{Environmentálna zodpovednosť:} Sledovanie uhlíkovej stopy, používania pesticídov a dodržiavania predpisov v oblasti dobrých životných podmienok zvierat.
\end{itemize}
\vspace{1em}

Hlavným prvkom slovenského systému by bola \textbf{povolená blockchainová sieť na báze konzorcia}. Povolený prístup zabezpečí, že sa môžu zúčastniť iba overené subjekty – ministerstvá, výrobcovia, spracovatelia a regulačné orgány – čím sa vyvažuje transparentnosť a kontrola údajov.

Každý oprávnený účastník by prevádzkoval uzol siete, čím by sa podieľal na overovaní transakcií a zabezpečoval odolnosť systému. Blockchainová infraštruktúra by sa integrovala s existujúcimi národnými poľnohospodárskymi databázami, colnými systémami a platformami EÚ, ako sú TRACES a EFSA.
\cite{Charlebois2024_Digital}.
\vspace{1em}

\textbf{Technický návrh} zahŕňa aj integráciu internetu vecí (IoT). Inteligentné senzory, RFID štítky a QR kódy by automaticky zaznamenávali údaje, ako sú teplota, vlhkosť a záznamy o preprave. Tieto zariadenia by sa prepájali s blockchainom prostredníctvom zabezpečených API, čím by sa minimalizovalo manuálne zadávanie údajov a riziko falšovania.
\vspace{1em}

Pre správu údajov by sa mala použiť hybridná architektúra: kritické údaje o sledovateľnosti (časové pečiatky, prevody vlastníctva, certifikáty) uložené v reťazci, zatiaľ čo veľké alebo citlivé súbory (laboratórne výsledky, obrázky, dokumenty) sú uchovávané v šifrovaných databázach mimo reťazca prepojených prostredníctvom hashov. To zaisťuje efektívnosť a škálovateľnosť a zároveň zachováva auditovateľnosť.
{Patel2023FoodSafety}
\vspace{1em}

Pilotný projekt by sa zameral najskôr na \textbf{sektor mlieka a mäsa}, ktoré majú vysoké nároky na sledovateľnosť a etablované exportné trhy. Účastníci – regionálne družstvá, bitúnky a distribútori – by testovali štandardizáciu formátov údajov a transakčných postupov. Každý účastník by prešiel digitálnym zaškolením.
\cite{2022SupplyChains}.

\begin{table}[h!] \centering \begin{tabular}{@{}cl@{}} \toprule \textbf{ } & \textbf{Popis hlavných krokov na digitálne zaškolenie} 
\\ \midrule 1 & Registrácia a overenie identity prostredníctvom národného poľnohospodárskeho registra. 
\\ 2 & Integrácia existujúcich digitálnych nástrojov (napr. ERP systémy, softvér na riadenie fariem) s blockchainovými API. 
\\ 3 & Školenie a certifikácia zamestnancov v protokoloch zadávania údajov. 
\\ 4 & Inštalácia IoT senzorov alebo systémov označovania QR kódmi. \\ \bottomrule \end{tabular} \caption{Proces digitálneho zaškolenia účastníka} \label{tab:zastrenie} 
\end{table}

Po validácii by zozbierané údaje tvorili základnú vrstvu blockchainovej siete sledovateľnosti, čím by vznikol prvý nemenný reťazec záznamov pre slovenskú produkciu potravín. Po vyhodnotení pilotného projektu by sa systém mohol rozšíriť aj do sektorov ovocia, zeleniny a nápojov.
\vspace{1em}

Každý nový sektor by mal prejsť analýzou rizík a definovaním kritických bodov sledovateľnosti podľa princípov HACCP. Pre každý kritický bod sa vytvorí záznamový protokol, ktorý bude zaznamenávať:
\begin{itemize}
\item Dátum a čas spracovania produktu,
\item Identifikáciu zdroja suroviny,
\item Použité chemické látky alebo hnojivá,
\item Parametre prepravy a skladovania,
\item Certifikáty kvality a laboratórne výsledky.
\end{itemize}
\cite{Kononets2022_SmartContracts}
\vspace{1em}

Integrácia nových sektorov by mala byť podporená automatizovanými nástrojmi, ako sú predkonfigurované šablóny záznamov a API pre prepojenie s existujúcimi ERP a SCM systémami. Tým sa minimalizuje potreba manuálneho zadávania údajov a zvyšuje presnosť dát.
\vspace{1em}

Prevádzka vyžaduje aj pravidelné bezpečnostné audity, testovanie škálovateľnosti a stresové testy, aby bola sieť schopná zvládnuť sezónne špičky, ako sú zbery úrody alebo vývozné obdobia.
\vspace{1em}

\subsection{Technologický rozvoj a inovácia} \label{Integrácia blockchainu do sledovania potravín}

Blockchainová infraštruktúra by mala byť navrhnutá s dôrazom na flexibilitu a modulárnosť, aby umožnila integráciu budúcich technológií:
\begin{itemize}
\item \textbf{Umelá inteligencia a prediktívna analytika:} Analýza veľkých dát z blockchainu umožní predpovedať riziká kontaminácie, optimalizovať logistiky a zlepšiť plánovanie pestovania.
\item \textbf{Smart kontrakty:} Automatizácia zmluvných podmienok, napr. platby dodávateľom po potvrdení dodania alebo certifikácie produktu.
\item \textbf{Interoperabilita s medzinárodnými platformami:} Prepojenie so systémami ako IBM Food Trust alebo VeChain umožní zdieľanie dát s exportnými partnermi a podporuje globálnu sledovateľnosť.
\item \textbf{Decentralizovaná identita (DID):} Každý účastník, zariadenie a produkt môže byť jednoznačne identifikovaný, čo zvyšuje bezpečnosť a znižuje riziko podvodov.
\end{itemize}

Tento technologický rámec vytvára základ pre inteligentné, dátovo orientované poľnohospodárstvo, ktoré podporuje udržateľnosť, efektívnosť a konkurenčnú výhodu Slovenska na európskom trhu.

\vspace{1em}

\subsection{Škálovanie a udržateľnosť systému} \label{scaling}

Po rozšírení na všetky prioritné sektory je potrebné pripraviť systém na dlhodobú udržateľnosť a škálovanie:
\begin{itemize}
\item \textbf{Postupné zapájanie menších farmárov a producentov:} Vytvorenie mobilných aplikácií a jednoduchých rozhraní na záznam údajov umožní aj malým subjektom bezpečne sa pripojiť k sieti.
\item \textbf{Financovanie a incentívy:} Kombinácia štátnych dotácií, európskych fondov a motivačných programov pre účastníkov zvyšuje adopciu systému.
\item \textbf{Priebežná aktualizácia technologických štandardov:} Blockchainový protokol, IoT zariadenia a API sa pravidelne aktualizujú podľa vývoja trhu a nových bezpečnostných štandardov.
\item \textbf{Monitoring dopadov na životné prostredie:} Systém umožňuje meranie uhlíkovej stopy a spotreby zdrojov, čím podporuje environmentálne ciele EÚ.
\end{itemize}

Takto implementovaný systém umožní postupný prechod od pilotného projektu k plne funkčnej, udržateľnej a interoperabilnej národnej platforme sledovateľnosti potravín.
\vspace{1em}

\section{Prípadové štúdie / praktické príklady} \label{dolezita}

V posledných rokoch čelí potravinový reťazec v mnohých ázijských krajinách narastajúcim výzvam: falšovanie potravín, nedostatočná sledovateľnosť pôvodu, nízka dôvera spotrebiteľa v bezpečnosti potravín a tlak na udržateľnosť. Dopyt spotrebiteľov po transparentnosti a autentickosti potravinových produktov rastie, zatiaľ čo regulačné požiadavky na bezpečnosť potravín sa sprísňujú. Táto štúdia sa zameriava na proces implementácie a integrácie technológie distribuovanej databázy (blockchain) v rámci potravinového reťazca v ázijsko‑pacifickom regióne – s cieľom analyzovať organizačné, technologické a procesné aspekty integrácie, výzvy, riešenia a dopady (nie podrobne technickú architektúru).
\vspace{1em}

Cieľom je identifikovať, ako organizácie v Ázii (farmári, spracovatelia, distribútori, maloobchodníci) zavádzajú blockchain riešenia na sledovanie potravín, aké kroky museli podniknúť, aké prekážky prekonali a aké prínosy či lekcie vyplývajú z takýchto projektov.
\vspace{1em}

Pri integrácii blockchainu pre sledovanie potravín v Ázii organizácie čelili rozhodnutiu, akú platformu a s akými partnermi budú spolupracovať. Kľúčové kritériá zahŕňali interoperabilitu s existujúcimi systémami (napr. ERP, IoT), náklady na implementáciu, škálovateľnosť riešenia a jednoduchosť adopcie pre rôzne typy účastníkov reťazca (farmári, spracovatelia, distribútori, maloobchod).
Napríklad spoločnosť Charoen Pokphand Foods Public Company Limited (CP Foods) z Thajska si vybrala partnerstvo s lokálnou agri‑tech firmou a zahraničnou technologickou spoločnosťou, aby implementovala blockchainový systém vo svojom dodávateľskom reťazci pre kuracie, bravčové, krevety a neskôr ďalšie produkty.
\cite{AsiaBlockchain}
\vspace{1em}

Kľúčové kroky vo výbere boli: identifikácia pilotného segmentu (napr. čerstvé mäso), výber partnerov z technológie sledovania a logistiky, definovanie interných procesov zberu dát (napr. pôvod suroviny, spracovanie, balenie), definovanie miery adopcie (farmári/bravčové farmy) a plán rozšírenia na celú organizáciu.
\vspace{1em}

Rovnako v prípade Vietnamu bola iniciatíva realizovaná formou partnerstva medzi technologickou firmou (The GrowHub) a investičnou organizáciou (Why Ventures), ktorá umožnila pilotné testovanie systému sledovania pre kakao a kávu – čo bol strategický krok na budovanie dôvery v exportnej agro‑výrobe.
\vspace{1em}

Po implementácii blockchainového riešenia v sledovanom reťazci Ázie boli identifikované nasledovné dopady:
\begin{itemize}
\item \textbf{Zlepšenie sledovateľnosti produktov:}
Organizácie boli schopné rýchlejšie a presnejšie identifikovať pôvod potraviny, spracovanie a distribúciu. Napríklad v prípade CP Foods spotrebiteľ mohol cez QR kód získať informáciu o farme, krmive, spracovaní, balení i transporte. 
\item \textbf{Rast dôvery spotrebiteľa a marketingový prínos:} Transparentnosť pôvodu sa stala marketingovým bodom, čo umožnilo firmám komunikovať vyššiu kvalitu a bezpečnosť produktu. V prípade Vietnamu farmárov spolupráca s blockchainovým partnerom umožnila zosúladiť sa s požiadavkami medzinárodných trhov (kakao, káva) a zvýšiť konkurenčnú výhodu. 
\item \textbf{Operatívna efektivita a zníženie rizík:}
Vďaka prehľadnosti reťazca bolo možné rýchlejšie reagovať na prípadné problémy, napríklad identifikovať mäso s rizikom kontaminácie či falšovania. Navyše zníženie chýb či strát vďaka lepšiemu manažmentu informácií.
\item \textbf{Ekonomické dopady pre účastníkov reťazca:} Predajné ceny mohli byť vyššie za produkty s garantovaným pôvodom, farmári získali prístup k vyšším hodnotám alebo novým trhom. Zároveň firmy mohli znížiť náklady spojené s krízovým manažmentom či spätnou výzvou produktov.
\end{itemize}

Celkovo sa zdôraznila potreba štandardizácie dát medzi účastníkmi reťazca, strategické plánovanie rozšírenia systému (napr. viac kategórií produktov), a budovanie infraštruktúry pre menšie podniky. Ďalej aj dôležitosť kultivácie partnerstva a dôvery v technológiu medzi všetkými účastníkmi.
\vspace{1em}

Implementácia v daných prípadoch ukázala, že hoci náklady a organizačné výzvy sú značné, prínosy v oblasti transparentnosti, spotrebiteľskej dôvery a konkurenčnej výhody sú pre firmy reálne
\vspace{1em}

Z analýzy vyplýva, že integrácia blockchainu do potravinového reťazca v Ázii fungovala najlepšie tam, kde bol prístup k pilotnému testovaniu, podpora zo strany organizácie, zapojenie partnerov od farmára po maloobchodníka a dávkový prístup – teda najskôr jedna kategória produktu, potom rozšírenie. Naproti tomu firmy, ktoré sa snažili zaviesť systém nárazovo pre celý portfólio alebo bez dostatočnej podpory partnerov, čelili vyššej miere odporu a komplikácií.



\section{Záver} \label{zaver} 

\clearpage

\bibliography{literatura}
\bibliographystyle{unsrt}

\end{document}