\documentclass[10pt,twoside,slovak,a4paper]{article}

\usepackage[slovak]{babel}
\usepackage[IL2]{fontenc}
\usepackage[utf8]{inputenc}
\usepackage{graphicx}
\usepackage{float}
\usepackage{url}
\usepackage{xurl}
\usepackage{hyperref} 
\usepackage{fancyhdr}

\usepackage{cite}

\pagestyle{headings}

\title{Blockchain pre sledovateľnosť potravín\thanks{Semestrálny projekt v predmete Metódy inžinierskej práce, ak. rok 2025/26, vedenie: Ivan Kapustík}} 

\author{Simona Kubicseková, Dominik Kužma, Maroš Kvietok \\[2pt]
	{\small Slovenská technická univerzita v Bratislave}\\
	{\small Fakulta informatiky a informačných technológií}\\
	}

\date{\small 30. september 2025}


\begin{document}

\maketitle

\begin{abstract}
Dodávateľské reťazce v agropotravinárskom priemysle sú zložité a zahŕňajú veľký počet aktérov. Spotrebitelia sa čoraz viac zaujímajú o pôvod a kvalitu potravín, čo vyvoláva tlak na producentov a distribútorov, aby poskytovali transparentné a overiteľné informácie. 

Táto štúdia sa zameriava na využitie technológie blockchain ako decentralizovaného rámca na zlepšenie sledovateľnosti potravín. Navrhovaný systém umožňuje monitorovanie potravinových výrobkov v reálnom čase od pôvodu až po spotrebu, čo by účinne obmedzovalo riziko a dopad kontaminácie potravín, pretože by umožnil rýchlu a presnú identifikáciu zdroja problému. 

Štúdia uplatňuje analytický prístup, ktorý kombinuje literárnu rešerš, návrh modelu blockchainového systému a prípadovú štúdiu z praxe agropotravinárskeho sektora. 

Očakávaným prínosom je zvýšenie dôvery spotrebiteľov, zníženie administratívnych nákladov a podpora udržateľných postupov v potravinárstve. Implementácia blockchainu môže tiež uľahčiť spoluprácu medzi aktérmi a zabezpečiť lepšiu transparentnosť v celom hodnotovom reťazci.
\end{abstract}

\clearpage
\tableofcontents
\thispagestyle{plain}

\clearpage
\pagestyle{plain}

\section{Úvod}

\section{Teoretický rámec} \label{teoria}

\subsection{Základné pojmy} \label{zakladne pojmy}

\subsection{Blockchain} \label{blockchain}

Technológia \textit{blockchain} predstavuje spôsob uchovávania dát v decentralizovanej sieti, ktorý eliminuje potrebu centrálneho sprostredkovateľa. Vznikla v roku 2008 ako súčasť návrhu kryptomeny Bitcoin, no odvtedy sa jej využitie rozšírilo do mnohých oblastí – od bankovníctva až po verejnú správu. 
\vspace{1em}

Blockchain možno v jednoduchej forme predstaviť ako reťaz blokov (ako môžeme vidieť na obrázku~\ref{fig:blockchain_architecture}), kde každý blok obsahuje súbor transakcií a odkaz (tzv. hash) na predchádzajúci blok. Tento mechanizmus zabezpečuje, že žiadny blok nemožno spätne zmeniť bez narušenia celej štruktúry reťazca. Takto sa vytvára dôveryhodný a nemenný záznam udalostí, ktorý je navyše distribuovaný medzi všetkými účastníkmi siete.

\begin{figure}[H]
    \centering
    \includegraphics[width=0.8\textwidth]{blockchain.png}
    \caption{Zjednodušená architektúra blockchainu}
    \label{fig:blockchain_architecture}
\end{figure}

Jednou z hlavných výhod blockchainu je jeho \textit{transparentnosť}. Každý účastník siete môže overiť údaje bez nutnosti dôvery v konkrétnu autoritu. Tento princíp je užitočný všade tam, kde je dôležité zabezpečiť integritu informácií – napríklad v logistike, energetike, či evidencii majetku. Naopak, nevýhodou býva energetická náročnosť niektorých mechanizmov a obmedzená rýchlosť spracovania transakcií.
\vspace{1em}

V posledných rokoch sa výskum zameriava na efektívnejšie modely, ako sú \textit{Proof of Stake} alebo \textit{Byzantine Fault Tolerance}, ktoré umožňujú rýchlejšie a ekologickejšie spracovanie údajov. Tieto prístupy robia blockchain atraktívnym aj pre oblasti mimo finančného sektora, napríklad pre zdravotníctvo či verejné registre \cite{zheng2018blockchain}.
\vspace{1em}

V súčasnosti sa blockchain skúma aj ako nástroj pre transparentnosť a dôveryhodnosť v rámci dodávateľských reťazcov, kde umožňuje bezpečné zdieľanie údajov medzi nezávislými aktérmi. Aj keď jeho implementácia prináša technické a organizačné výzvy, je považovaný za perspektívnu technológiu s potenciálom zásadne zmeniť spôsob, akým pracujeme s dátami.


\subsection{Sledovateľnosť potravín (Food Traceability)} \label{sledovatelnost}

Na poznatkoch o fungovaní technológie \textit{blockchain} je možné ďalej stavať pri jej uplatnení v konkrétnych oblastiach, kde je transparentnosť a dôveryhodnosť údajov mimoriadne dôležitá. Jednou z takýchto oblastí je \textit{sledovateľnosť potravín} (\textit{food traceability}), ktorá sa v súčasnosti stáva jednou z kľúčových tém v oblasti potravinovej bezpečnosti a riadenia dodávateľských reťazcov.
\vspace{1em}

Globalizácia a technologický pokrok spôsobili, že potraviny prechádzajú cez množstvo aktérov, hraníc a procesov, čo výrazne zvyšuje riziko podvodov, neefektívnosti a ohrozenia kvality. Tradičné systémy, ktoré sa spoliehajú na papierové záznamy, čiarové kódy alebo fragmentované databázy, často nedokážu zabezpečiť dostatočnú transparentnosť a spoľahlivosť. V tejto situácii sa \textit{blockchain} javí ako perspektívna technológia, ktorá môže zásadne zmeniť fungovanie potravinových dodávateľských reťazcov. Jeho základné vlastnosti – nemennosť, transparentnosť a odolnosť voči manipulácii – umožňujú vytvoriť systém, v ktorom je možné sledovať cestu potraviny od pôvodu až po konečného spotrebiteľa s vysokou mierou dôveryhodnosti \cite{galvez2018blockchain}.
\vspace{1em}

V posledných rokoch sa koncept sledovateľnosti potravín rýchlo rozvíja, keďže rastie dopyt po bezpečných a overiteľných produktoch. Viaceré podniky a obchodné reťazce implementujú digitálne riešenia, ktoré umožňujú presne určiť pôvod a kvalitu potravín. Napríklad využitie blockchainu v potravinovom sektore dokázalo výrazne skrátiť čas potrebný na vysledovanie pôvodu produktu z niekoľkých dní na niekoľko sekúnd. Podobné systémy testujú aj ďalšie svetové spoločnosti, ktoré sa snažia zvýšiť dôveru spotrebiteľov a eliminovať falšovanie potravín. Tieto príklady potvrdzujú, že blockchain nie je len teoretickým konceptom, ale aj praktickým nástrojom, ktorý dokáže zlepšiť rýchlosť a spoľahlivosť vysledovateľnosti \cite{tian2017food}.
\vspace{1em}

V rámci moderných riešení sa blockchain často kombinuje s technológiami Industry~4.0 a Web~3.0. Ide najmä o internet vecí (IoT), RFID, QR kódy, cloud computing a GPS. Tieto technológie umožňujú zber a prenos údajov v reálnom čase, ktoré sú následne zabezpečené a zdieľané prostredníctvom blockchainu. Zatiaľ čo niektoré pokročilé nástroje, ako umelá inteligencia, analýza veľkých dát, \textit{edge computing}, IPFS či metaverzum, sa zatiaľ používajú len zriedka, práve ich prepojenie s blockchainom predstavuje potenciál pre sofistikovanejšie a efektívnejšie riešenia sledovateľnosti.
\vspace{1em}

Zaujímavý smer predstavujú aj návrhy inteligentných systémov, ktoré kombinujú blockchain s umelou inteligenciou, cloudom a \textit{big data} na znižovanie potravinového odpadu. Takéto prístupy ukazujú, že sledovateľnosť nemusí byť len o spätnej kontrole pôvodu, ale aj o aktívnom riadení tokov potravín, predchádzaní stratám a efektívnejšom využívaní zdrojov. V budúcnosti môžu podobné modely umožniť rýchlejšie reakcie na krízy, lepšiu udržateľnosť a optimalizáciu celého dodávateľského reťazca.
\vspace{1em}

Tieto princípy majú paralely aj v iných odvetviach, kde sa blockchain využíva na zvýšenie dôvery a automatizáciu procesov, napríklad vo financiách či logistike. Spoločným znakom je využitie distribuovaného registra a \textit{smart kontraktov}, ktoré zabezpečujú transparentnosť a automatické vynucovanie pravidiel. V potravinovom sektore to znamená, že každý krok – od farmára cez spracovateľa až po obchodníka – môže byť zaznamenaný a overený bez možnosti dodatočnej manipulácie.
\vspace{1em}

Na začiatku tohto projektu sa sledovateľnosť potravín chápe ako súbor technológií a procesov, ktoré umožňujú prepojiť údaje o pôvode, spracovaní a distribúcii potravín do jednotného, dôveryhodného systému. Cieľom je zvýšiť transparentnosť, udržateľnosť a bezpečnosť potravinového reťazca, pričom blockchain môže slúžiť ako technický základ pre tieto ciele.

\section{Integrácia blockchainu do sledovania potravín} \label{ina}


\section{Prípadové štúdie / praktické príklady} \label{dolezita}


\section{Záver} \label{zaver} 

\bibliography{literatura}
\bibliographystyle{plain}

\end{document}
